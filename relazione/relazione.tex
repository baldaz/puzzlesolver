\documentclass[10pt,a4paper]{article}
\usepackage[utf8]{inputenc}
\usepackage{alltt}
\usepackage{caption}
\usepackage{listings}
\usepackage{xcolor}
\usepackage{graphicx}
\usepackage[T1]{fontenc}
\usepackage{lmodern}

\title{Programmazione concorrente e distribuita 2014-2015. Prima parte}
\author{Andrea Giacomo Baldan 579117}

\begin{document}
\maketitle
\begingroup
\let\clearpage\relax
\section*{ORGANIZZAZIONE E SCELTE IMPLEMENTATIVE}
\section*{ALGORITMO DI RICOSTRUZIONE}
L'algoritmo di risoluzione implementato nella classe SortAlgSeq, derivata dalla classe base astratta SortAlg, è appunto un algoritmo di risoluzione sequenziale, e si può riassumere in 3 passi:
\begin{enumerate}
    \item
    Localizzazione del primo pezzo del puzzle, che corrisponde al pezzo avente \lq\lq VUOTO\rq\rq a nord e ad ovest.
    \item
    Inizio ciclo: ordina la riga a partire dal primo pezzo
    \item
    Ricerca del pezzo a sud del primo pezzo della nuova riga ora ordinata, se il pezzo a sud non esiste e troviamo dunque \lq\lq VUOTO\rq\rq, il ciclo si ferma in quanto tutte le righe sono quindi già state ordinate, altrimenti si ripete (1) utilizzando il \lq\lq nuovo\rq\rq primo pezzo.
\end{enumerate}
\endgroup
\end{document}