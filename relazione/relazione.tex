% Created 2015-08-11 mar 20:16
\documentclass[11pt,a4paper]{article}
\usepackage[utf8]{inputenc}
\usepackage{alltt}
\usepackage{caption}
\usepackage{hyperref}
\usepackage{listings}
\usepackage{xcolor}
\usepackage{graphicx}
\usepackage{lmodern}
\usepackage[top=2in, bottom=1.5in, left=0.7in, right=0.7in]{geometry}
\DeclareCaptionFormat{listing}{\rule{\dimexpr\textwidth+17pt\relax}{0.4}\vskip1pt#1#2#3}
\captionsetup[lstlisting]{singlelinecheck=false, margin=0pt,
	font={bf,footnotesize}}
               

\usepackage{titlesec}
\titleformat{\section}{\normalfont\Large\bfseries}{\thesection}{1em}{}[{\titlerule[0.8pt]}]
\usepackage[T1]{fontenc}
\usepackage{libertine}
\renewcommand*\oldstylenums[1]{{\fontfamily{fxlj}\selectfont #1}}
\definecolor{wine-stain}{rgb}{0.5,0,0}
\hypersetup{colorlinks, linkcolor=wine-stain, linktoc=all}
\usepackage{lmodern}
\lstset{basicstyle=\normalfont\ttfamily\small,numberstyle=\small,breaklines=true,frame=tb,tabsize=1,showstringspaces=false,numbers=left,commentstyle=\color{grey},keywordstyle=\color{black}\bfseries,stringstyle=\color{red}}
\newenvironment{changemargin}[2]{\list{}{\rightmargin#2\leftmargin#1\parsep=0pt\topsep=0pt\partopsep=0pt}\item[]}{\endlist}
\newenvironment{indentmore}{\begin{changemargin}{1cm}{0cm}}{\end{changemargin}}
\author{Andrea Giacomo Baldan 579117}
\date{\today}
\title{Programmazione Concorrente e Distribuita 2014-2015. Prima parte.}
\hypersetup{
  pdfkeywords={},
  pdfsubject={},
  pdfcreator={Emacs 24.3.1 (Org mode 8.2.10)}}
\begin{document}

\maketitle
\tableofcontents

\section{Organizzazione e scelte implementative}
\label{sec-1}
Ho scelto di seguire un pattern MVC per la modularità e la separazione logica - output che offre, anche in assenza di interfaccia grafica, nel caso futuro
in cui venisse implementata, il design pattern scelto agevolerebbe notevolmente il lavoro.
\subsection{Organizzazione delle classi}
\label{sec-1-1}
Le classi sono raggruppate in un package, \textbf{puzzlesolver} e all'esterno la classe \textbf{PuzzleSolver} contiene il \verb~main~ e le chiamate esecutive del programma.
\section{Principi di OOP}
\label{sec-2}
\subsection{Information hiding}
\label{sec-2-1}
I campi dati delle classi sono stati dichiarati tutti \verb~private~, accedibili mediante classici metodi \verb~getters~ e \verb~setters~.
\section{Algoritmo di ricostruzione}
\label{sec-3}
L'algoritmo di risoluzione implementato nella classe SortAlgSeq, derivata dalla classe base astratta SortAlg, è appunto un algoritmo di risoluzione 
sequenziale, e si può riassumere in 3 passi:
\begin{enumerate}
\item Localizzazione del primo pezzo del puzzle, che corrisponde al pezzo avente "VUOTO" a nord e ad ovest.
\item Inizio ciclo: ordina la riga a partire dal primo pezzo
\item Ricerca del pezzo a sud del primo pezzo della nuova riga ora ordinata, se il pezzo a sud non esiste e troviamo dunque "VUOTO", il ciclo si ferma 
in quanto tutte le righe sono quindi già state ordinate, altrimenti si ripete (1) utilizzando il "nuovo" primo pezzo.
\end{enumerate}
\section{Note}
\label{sec-4}
Il progetto è stato sviluppato in ambiente linux, utilizzando la JVM versione 1.7.0.
% Emacs 24.3.1 (Org mode 8.2.10)
\end{document}